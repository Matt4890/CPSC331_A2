\documentclass[a4paper, 12pt, titlepage]{article}

\def\code#1{\texttt{#1}}
\usepackage{listings}
\usepackage{color}
\usepackage{amsmath}
\usepackage{indentfirst}

\definecolor{dkgreen}{rgb}{0,0.6,0}
\definecolor{gray}{rgb}{0.5,0.5,0.5}
\definecolor{mauve}{rgb}{0.58,0,0.82}

\lstset{
  frame=tb,
  language=Java,
  aboveskip=3mm,
  belowskip=3mm,
  showstringspaces=false,
  columns=flexible,
  basicstyle={\small\ttfamily},
  numbers=left,
  numberstyle=\tiny\color{gray},
  keywordstyle=\color{blue},
  commentstyle=\color{dkgreen},
  stringstyle=\color{mauve},
  breaklines=true,
  breakatwhitespace=true,
  tabsize=3
}

% Basic document details
\title{CPSC 331: Assignment 2}
\author{
  Matthew Allwright\\
  \texttt{30037812}
  \and
  Karim Beyk\\
  \texttt{30027342}
  \and
  Seth Campbell\\
  \texttt{10152719}
}

\begin{document}

\maketitle

\section*{AVL Tree Defenition and Properties}

\subsection*{Question 1: Prove that $s_i \geq F_{i+1}+F_{i+2}-1$.}

Induction and fun mathematics.

\section*{Insertions}

\subsection*{Question 2: Briefly explain why the only nodes in a tree whose heights or balance factors might
have changed lie on the path from the a leaf up to the root of the tree.} 

Both the height and the balance factor of a node are exclusively determined by the connected nodes of a greater depth.

\subsection*{Question 3: Briefly explain why the balance factors of the nodes on the above path are all either $2$, $1$, $0$, $-1$, or $-2$.} 

Assuming the tree was indeed an AVL tree before the insertion, 
then all nodes of the tree must have a balance factor within $\{1, 0, -1\}$. 
By inserting a node, 
the balance factor of any preceeding node will be modified by a value of either $1$ or $-1$, 
as only one node was inserted. 
Thus, 
the extreme values of the set of possible balance factors may become more ``extreme'' by an absolute value of 1, 
meaning $-1$ may become $-2$ and $1$ may become $2$.
Plus the other values are still attainable.

\subsection*{Question 4: Explain why if $v$ is some node on this path in the tree, and the height of $v$ has not been changed, then the heights and balanced factors of all the nodes on the path that are above $v$ have not been changed either.} 

Height: 
The height of a node depends on the maximum height of its children. 
If neither of the node's children's heights have changed, 
then the node's own height cannot have changed.

Balance Factor: 
The balance factor of a node is defined as the height of the left child minus the height of the right child. 
If neither of the node's children's balance factors have changed, 
then the node's own balance factor cannot have changed.

\subsection*{Question 5: Comparing Figures 5 and 6, confirm that node $\alpha$ has height h and balance factor $0$ after the described rotation.} 

After the rotation, 
node $\alpha$ has $T_2$ as its left child, 
and $T_3$ as its right child. 
No nodes within $T_2$ or $T_3$ have been changed, 
and thus they each maintain the height of $h-1$, 
as they did before the rotation. 
Because each child of $\alpha$ has the same height, 
the balance factor of $\alpha$ is $0$. 
Also, 
the maximum height of either child of $\alpha$ is $h-1$, 
meaning that the height of $\alpha$ is indeed $h$.

\subsection*{Question 6: Confirm that the node $\beta$ has height $h+1$ and balance factor $0$ as well after the described rotation.} 

Something something copy question 5.

\subsection*{Question 7: Explain why the binary search tree is once again an AVL tree after the described rotation.} 

\subsection*{Question 8: Confirm that $\alpha$ and $\beta$ each have a balance factor within $\{−1, 0, 1\}$ and height $h$ after this adjustment.} 

\subsection*{Question 9: Explain why $\gamma$ has height $h+1$ and balance factor $0$ after this operation. Using this, explain why the resulting tree is an AVL tree after this operation.} 

\subsection*{Question 10: Consider the case that $\alpha$ has balance factor $-2$, so that the subtree with root $\alpha$ is as shown in Figure 10. Describe two more cases that should probably be called the \textit{right-right} case and the \textit{right-left} case that might arise, corresponding this one, and describe the adjustments that can be used to produce AVL trees when these cases arise.} 

\section*{Deletions}

\subsection*{Question 11: Suppose that you perform a right rotation at $\alpha$ so that the subtree with root $\beta$ after this adjustment is as shown in Figure 6. Briefly explain why $\alpha$ has height $h+1$ and balance factor $1$ after this adjustment.} 

\subsection*{Question 12: Briefly explain why $\beta$ has height $h+2$ and balance factor $-1$ after this adjustment.} 

\subsection*{Question 13: Briefly explain why the entire binary search tree is an AVL tree after this adjustment.} 

\subsection*{Question 14: Now consider the case that $\alpha$ has balance factor $-2$, instead, so that the subtree with root $\alpha$ is as shown in Figure 10. Briefly describe another three cases corresponding to this, along with the adjustments that should be made for each.} 

\section*{Implementing an AVL Tree}

\subsection*{Question 15: Briefly describe the structure of an algorithm for an insertion into an AVL tree.} 

\subsection*{Question 16: Briefly describe the structure of an algorithm for a deletion from an AVL tree.} 

\subsection*{Question 17: Provide a complete \code{AVLDictionary.java} class for assessment.} 

\end{document}
